\chapter{Introduzione}
Con questa tesi si ha come obbiettivo di studiare e analizzare le varie tecniche di spam detection ed in particolare analizzare le tecniche online. In particolare le tecninche vengono classificate sulla base dei segnali che utilizzano. Il fattore chiave è che non ci sono, o meglio sono poche, al momento tecniche online di spam detection, ovvero tecninche che rilevano lo spam durante la fase di crawling. Infatti quasi tutti i metodi tentano di fare il crawling dell'intero web e successivamente classificare le pagine in spam o buone. 
Introduzione alla tesi
Perché viene fatta la tesi
Dove viene fatta
Problemi relativi
Soluzini
\section{Web spam}
Con il termine web spamming si fa riferimento a tutti i metodi che tentano di manipolare gli algoritmi di ranking dei motori di ricerca per aumentare il valore di ranking di alcune pagine rispetto ad altre\cite{ilprints646}.
Dato il numero esorbitante di pagine che vengono create e pubblicate sul web, gli utenti competono per poter far comparire le proprie pagine tra le prime dei risultati di una query. %introdurre statistiche su quante persone visitano i primi link e la percentuale di spam descritta in survey
Il fenomedo dello spamming o spamindexing comporta che la qualità delle ricerche decrementa, gli utenti tenderanno ad utilizzare altri motori di ricerca, l'indicizzazione di pagine che non sono utili e l'aumento del costo delle operazioni di query ed infine la causa di malware e reindirizzamento verso contenuto per adulti\cite{Spirin:2012:SWS:2207243.2207252}.