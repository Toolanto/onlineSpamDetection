\chapter{Dataset e strumenti utilizzati}
\lstset{basicstyle=\small\ttfamily,keywordstyle=\color{black}\bfseries,commentstyle=\color{darkgray},stringstyle=\color{black},showstringspaces=true} 
Questo capitolo illustra il dataset utilizzato per l'implementazione e l'esecuzione dei test (che verranno discussi nel capitolo successivo) e i vari strumenti (come i framework e librerie) utilizzate per rendere più efficienti i vari test.
\section{Descrizione del dataset}
Il dataset utilizzato per gli espirementi è WEBSPAM-UK2007 \cite{webspam-uk2007} ottenuto da un crawl del dominio ``.uk'' nel maggio del 2007, grazie al supporto del ``Laboratorio di Algoritmica per il Web'' dell'Università degli Studi di Milano; è composto da un insieme di 105.896.555 pagine che fanno parte di 114.529 host. I dati di questo dataset, rappresentati degli host, sono stati etichiettati in spam o non spam per mezzo di alcuni volontari. Il dataset è composto da tre parti:
\begin{itemize}
 \item Le etichette associate agli host divise in due gruppi. Il primo gruppo è costituito da 2/3 dell'insieme degli host valutati, è fornito per essere come dataset di traning ed è costituito da 3776 nodi etichettati come non spam, 222 nodi etichettati spam ed infine 277 nodi etichettati ``undecided'' ovvero nodi di cui la natura (spam o non spam) è in dubbio. Il secondo gruppo di etichette contiene il rimanente 1/3 dell'insieme degli host valutati, è fornito per essere utilizzato come dataset di testing ed è costituito da 1933 nodi etichettati non spam, 122 nodi etichettati spam e 149 etichettati ``undecided''.\\
 Le etichette sono associate in due fasi. Nella prima fase l'esaminatore ha a diposizione solo il contenuto della collezione e la lista degli host collegati per ogni host valutato. Nella seconda fase, invece, ha accesso all'etichette, per un nodo, assegnate da altri utenti e gli viene richiesto di rivalutare le sue valutazioni se è in disaccordo con altri. Quindi viene realizzato un file che contiene tali informazioni.\\ 
 Gli host sono identificati attraverso un \textit{id} che va da 0 a 114.528 a cui è associato l'URL corrispondente. Quindi un file specifica per ogni \textit{id} un'etichetta tra ``nonspam'',``spam'' e ``undecided'', oltre all'etichetta vengono associate le singole valutazioni effettuate dagli utenti e la media risultante delle valutazioni effettuate per quell'host tenendo conto che l'etichetta spam è uguale a 1, l'etichetta non spam è uguale a 0 e l'etichetta di ``undecided'' ha valore 0.5. Un esempio è rappresentato di seguito: 
 \begin{lstlisting}[frame=trbl,postbreak=\space, breakindent=5pt, breaklines]
ID ETICHETTA MEDIA VALUTAZIONI
5 nonspam 0.00000 j1:N,j2:N
100 nonspam 0.33333 j14:N,j17:S,j7:N
120 spam 1.00000 j18:U,j4:S
170 undecided - j13:U,j20:U
210 undecided 0.50000 j15:N,j16:S,j22:U
\end{lstlisting}
Nell'esempio vengono rappresentati 5 host; prendendo in cosiderazione l'host con \textit{id} uguale a 5 si nota subito che è stato giudicato come nodo non spam e che la media delle due votazioni è uguale a 0 visto che entrambi gli utenti hanno giudicato il nodo non spam (es. j1:N; N indica non spam, S indica spam e U indica che la natura del nodo non è definita).\\ 
Per i test tutte queste informazioni sono superfle, infatti sono stati utilizzati solo i dati relativi alle etichette associate ai nodi in modo d'avere l'insieme dei nosi spam e l'insieme dei nodi non spam.

 
 \item Il grafo definito con due differenti livelli di granularità: un è a livello di host mentre un secondo grafo è definito a livello di pagine. Il grafo a livello di host (che poi sarà quello utilizzato per eseguire i test) è formato da 114.528 nodi che rappresentano gli host mentre il secondo che è formato  105.896.555 di nodi che rappresentano le pagine web che sono connesse da circa 3.7 miliardi di archi ovvero di link tra le pagine.\\
 Il grafo degli host quindi rappresenta i link tra gli URL mappando vari link tra le pagine in differenti host in un singolo link a cui è associato un peso. Il file è composto dalla prima riga che indica il numero di host nel grafo e le successive righe che contengono gli \textit{outlink} degli host, ad esempio alla riga due verranno indicati gli \textit{outlink} del nodo 0 alla riga 3 gli \textit{outlink} del nodo 1 e cosi via. Dato che i link sono pesati la sintassi completa per ogni riga che rappressenta un nodo è:
 $$
 dest_1:nlinks_1,dest_2:nlinks_2,...,dest_k:nlinks_k
 $$
 dove \(dest\) è il nodo di arrivo del link ed è rappresentato dall'\textit{id} del nodo di destinazione mentre \(nlinks\) è il numero di link tra le pagine dei due host.
 Ad esempio di seguito è illustrato una porzione di un grafo degli host:
\begin{lstlisting}[frame=trbl,postbreak=\space, breakindent=5pt, breaklines]
114529
1005:1 8306:2 9596:1 
8107:3320 22950:4 108053:1
24003:1
24003:1
...
\end{lstlisting}
Nella prima riga dell'esempio viene specificato che il grafo è composto da 114529 nodi. Nella seconda riga si nota che il nodo 0 ha 3 outlink, uno verso il nodo 1005, un altro link verso il nodo 9596 ed infine due link verso il nodo 8306. La terza riga indica che il nodo 1 ha 3 outlink: un link verso il nodo 8107 con un peso 3320 (quindi le pagine del nodo 1 hanno 3320 link verso le pagine del nodo8107), quattro link verso il nodo 22950 ed infine un link verso il nodo 108053. Il file quindi sarà lungo 114530 righe.\\
%nel nostro caso non sono state usate le informazioni sui pesi degli archi si è scelto di eliminare tale informazione
Quindi utilizzando queste informazioni si può convertire il file che descrive il grafo in  in modo da poter essere successivamente manipolato tramite il framework WebGraph \cite{Boldi03thewebgraph}.
%si può inserire come fare la conversione
\item Il contenuto delle pagine HTML fornito in formato WARC. ``Il formato WARC specifica un metodo per combinare molte risorse in un unico file con le relative informazioni'' (\url{http://www.digitalpreservation.gov/formats/fdd/fdd000236.shtml} il 5/5/2014).
\end{itemize}

Quindi per lo sviluppo dei test sono state utilizzate due informazioni utili derivate dal dataset: le etichette che identificano i nodi del grafo spam da quelli che sono non spam e la struttura del grafo che verrà manipolata tramite l'utilizzo del framework WebGraph \cite{Boldi03thewebgraph}. WebGraph è un framework per la compressione di enormi grafi e permette, quindi, di eseguire operazioni sul grafo in maniera molto semplice e veloce. 

\section{La Tau di Kendall}
Uno degli strumenti utilizzati per il confronto di due vettori di rank (che saranno i vettori di \textit{trustrank} e  di \textit{anti-trust rank}) è la Tua di Kendall \cite{KendallTau}. Più precisamente la Tau di Kendall valuta il grado di similarità tra due vettori di rank ovvero dati due vettori \(r_i\) e \(s_i\) con \(i=0,1,...,n-1\) e \(i<j\), si dice che una coppia \((i,j)\) è:
\begin{itemize}
 \item \textit{concordante} se \(r_i-r_j\) e \(s_i-s_j\) sono entrambe diverse da 0 e hanno lo stesso segno;
 \item \textit{discordante} se \(r_i-r_j\) e \(s_i-s_j\) sono entrambe diverse da 0 e hanno segno opposto;
 \item \textit{r-tie} se \(r_i-r_j=0\);
 \item \textit{s-tie} se \(s_i-s_j=0\);
 \item \textit{joint tie} se \(r_i-r_j=0\) e \(s_i-s_j=0\).
\end{itemize}
Allora viene definito  \(C\) come il numero delle coppie concordanti, \(D\) il numero delle coppie discordanti, \(T_r\) il numero delle coppie r-tie, \(T_s\) il numero delle coppie s-tie ,\(J\) il numero delle coppie joint tie e \(N=n(n-1)/2\). Quindi \(C+D+T_r+T_s-J = N\). La Tau di Kendall può essere calcolata come:
\begin{equation}
 \tau = \frac{(C-D)}{[(N-T_r)(N-T_S)]^{1/2}}
\end{equation}
Per eseguire i test è stato scelto di usare la libreria  ``it.unimi.dsi.law'' \cite{libreriaLaw} fornita dal LAW (Laboratori di Algoritmica per il Web) dell'Università degli studi di Milano. Questa libreria contiene dei metodi che consentono di calcolare efficientemente la Tau di Kendall.

\section{Il framework Webgraph}
Per la gestione del grafo è stato usato il framework WebGraph \cite{Boldi03thewebgraph}. Questo framework permette la compressione dei grafi e di eseguire delle operazioni su di essi in maniera molto efficiente. Noi utiliziamo solo una piccola parte delle risorse che mette a disposizine WebGraph. Infatti esso incorpora:
\begin{itemize}
 \item un insieme di codici di compressione per la memorizzazione dei grafi;
 \item algoritmi che permettono di accedere al grafo compresso senza ogni volta doverlo decomprimere;
 \item algoritmi di analisi dei grafi;
 \item insiemi di dati che si possono analizzare.
\end{itemize}
Dal momento che i nostri test vanno ad esaminare delle proprietà di alcuni algoritmi sui grafi, questo framework rende le operazioni molto efficienti.

