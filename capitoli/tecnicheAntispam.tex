\chapter{Tecniche di spam detection}
Il capitolo illustra le tecniche per spam detection presenti in letteratura al momento in cui si sta scrivendo. Le tecniche verrano divise sulla base del tipo di segnali che usano: contenuto, grafo ottenuto dalla fase di crawling, altri segnali (es. header delle richieste HTTP). Nella prima parte del capitolo verranno illustrate le tecniche basate sul contenuto, nella parta centrale le tecniche basate su grafo ed infine le tecniche che fanno uso di altri tipi di segnali.

\section{Tecniche basate sul contenuto}
Uno dei primi studi sul web spam è descritto in \cite{Fetterly:2004:SDS:1017074.1017077}, in questo articolo vengono eseguite delle analisi statistiche che dimostrano che le  pagine spam  hanno proprietà diverse rispetto alle pagine con contenuti. Queste proprietà sono classificate come segue:
\begin{itemize}
 \item 	Proprietà degli URL: il link spam è una particolare forma di spam dove gli spammer cercano di aumentare il rank derivato da algoritmi basati sulla struttura dei link. Uno spammer cerca di creare automaticamente delle pagine di bassa qualità che puntano a una pagina target \textit{p}. In questo articolo viene fatta un'analisi sulle proprietà dei link ed è stato trovato che le caratteristiche degli URL di un HOST sono indizzi per identificare lo spam. In particolare è stato trovato che il nome di un HOST con molti caratteri, punti, barre e numeri è un buon indicatore di spam. Un modo semplice per classificare le pagine sarebbe quello di usare un soglia. % immagine grafico
 \item Host name resolution: Alcuni motori di ricerca (Google) data una query q, danno un rank più alto a un URL u se i componenti dell'host di u conentengo i termini della query. Gli spammers per sfruttare questo popolano le pagine con le URL le cui componenti contengono query popolari che sono rilevanti per un certo settore e impostano un DNS per risolvere questi host name. Spammers creano un grande numero di nomi per avere un alto rank per una grande varietà di query. Per determinare questa forma di spam basta vedere quanti nomi vengono risolti con lo stesso indirizzo.


Una altra caratteristica è che Google nella sua variante di PageRank assegna un peso maggiore a link esterni e sempre di più se ci sono molti link che puntano a differenti pagine di un altro server. Spammer cercano di popolare le pagine con molti link ad altri host tipicamente questi risolvono uno o pochi indirizzi. Per rilevare lo spam in viene calcolato la media dell' host-machine-ratio. L'host-machine-ratio di una pagina è definita come la grandezza degli host name riferiti dai link all'interno della pagina diviso per la dimensione dell'insieme degli indirizzi distinti con cui gli host name vengono risolti. L'host-machine-ratio è la media degli host-machine-ratio di tutte le pagine di un host. Un host-machine-ratio alto indica che questa ci sono molti riferimenti a host differenti è questa è un indicazione di spam.
 \item Proprietà del contenuto: le pagine generate automaticamente hanno tutte lo stesso template, in particolare ci sono numerosi siti di spam che dinamicamente generano pagine che hanno uno stesso numero di parole.
 \item Proprietà clustering: una tecninca per determinare lo spam è quella di clusterizzare le pagine in base alla somiglianza dei template. Visto che le pagine di spam sono molto simili tra loro, identiicando molte pagine con la stessa struttura è probabile che siano di spam.
\end{itemize}
