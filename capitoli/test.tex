\chapter{Test dei metodi di spam detection}
\lstset{basicstyle=\small\ttfamily,keywordstyle=\color{black}\bfseries,commentstyle=\color{darkgray},stringstyle=\color{black},showstringspaces=true} 
Nei capitoli precendenti sono stati illustrati vari metodi che di spam detection classificati basandosi sui segnali in ingresso che essi hanno bisogno per poter identificare le pagine web; quindi si hanno tre classi: metodi basati sul contenuto, metodi basati sul grafo e metodi che utilizzano segnali diversi dai primi due. Tra questi metodi ne sono stati presi in esame due: \textit{Trustrank} e \textit{Anti-trust rank}. 

Si è scelto, quindi, di valutare l'efficacia di due  algoritmi \textit{linked base} (\textit{Trustrank} e \textit{Anti-trust rank}) se essi operassero in modo online. Più precisamente i vari test consistono nel verificare quanto questi due algoritmi di tipo offline riescano ad approssimare il loro comportamento se li facessimo operare in modo online ovvero durante la fase di crawling. Le domande che ci poniamo eseguendo questi test su i due algoritmi di spam dection offline (\textit{Trustrank} e \textit{Anti-trus rank}) sono:
\begin{itemize}
 \item possono questi algoritmi essere in grado di operare in modalità online?
 \item durante l'esecuzione in modalità online, quanto riescono ad approssimare il loro comportamento offline?
 \item è conveniente utilizzare questi algoritmi in modalità online?
\end{itemize}

E' doveroso specificare che un algoritmo di spam detection lavora offline se questo viene eseguito dopo l'attività di crawling (e quindi dopo che si ha a disposizione l'intero grafo ottenuto dai collegamenti tra le pagine), mentre un algoritmo di spam detection lavora online se questo viene eseguito durante il processo di crawling e quindi riesce a determinare all'istante se una pagina deve essere considerata spam o non spam. Dal momento che si è scelto di esaminare degli algoritmi \textit{linked base}, e sapendo che questi formulano delle conclusioni sulla natura delle pagine (ovvero se sono spam o non spam) esaminando la struttura dell'intero grafo, è interessante notare come questi si comportino se il grafo su cui fare le valutazioni è incompleto.
%altre considerazioni

Il capitolo è diviso nel seguente modo: nella prima parte verrà illustrato come è stato scelto di simulare il crawler per poter eseguire gli algoritmi offline duratne la fase di crawling; nella seconda parte verrà spiegato come sono stati implementati i test; infine verrano illustrati tutti i test.

\section{Simulazione del crawler}
Per simulare il comportamento di crawler, e quindi eseguire gli algoritmi di \textit{Trustrank} e \textit{Anti-trust rank}, abbiamo implementato una semplice visita sul grafo in ampiezza ovvero una BFS \cite{bfsCormen} (Breadth-First Search). La visita in ampiezza dato un grafo \(G=(V,E)\), dove \(V\) è l'insieme dei vertici del grafo ed \(E\) l'insieme degli archi del grafo, e un vertice \(s\) da cui far partire la visita, scopre tutti i vertici che sono raggiungibili da \(s\). La visita in ampiezza scopre tutti i vertici che si trovano a distanza \(f\) dal vertice di partenza e successivamente scopre i vertici che si trovano a una distanza successiva \(f+1\). In sostanza dato il nodo di partenza \(s\), la visita in ampiezza, scopre tutti i nodi vicini al nodo \(s\) e successivamente per ogni nodo vicino scoperto trova i vicini che non sono ancora stati visitati; questo processo viene iterato finche tutti i nodi del grafo raggiungibili da \(s\) sono visitati. 







