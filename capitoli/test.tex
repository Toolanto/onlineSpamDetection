\chapter{Test dei metodi di spam detection}
\lstset{basicstyle=\small\ttfamily,keywordstyle=\color{black}\bfseries,commentstyle=\color{darkgray},stringstyle=\color{black},showstringspaces=true} 
Nei capitoli precendenti sono stati illustrati vari metodi che di spam detection classificati basandosi sui segnali in ingresso che essi hanno bisogno per poter identificare le pagine web; quindi si hanno tre classi: metodi basati sul contenuto, metodi basati sul grafo e metodi che utilizzano segnali diversi dai primi due. Tra questi metodi ne sono stati presi in esame due: \textit{Trustrank} e \textit{Anti-trust rank}. 

Si è scelto, quindi, di valutare l'efficacia degli algoritmi di \textit{Trustrank} e \textit{Anti-trust rank} se essi operassero in modo online. Più precisamente i vari test consistono nel verificare quanto questi due algoritmi di tipo offline riescano ad approssimare il loro comportamento se li facessimo operare in modo online ovvero durante la fase di crawling.


