\chapter{Conclusioni}

L'obbiettivo di questa tesi è di analizzare delle tecniche online di spam detection per poter essere, successivamente, implementate in un web crawler. Tale lavoro risulterà utile nella progettazione di un modulo di spam detection all'interno del  web crawler \textit{BUbiNG}, sviluppato dal Laboratorio di Algoritmica per il Web dell'Università degli Studi di Milano.

Per raggiungere tale obbiettivo, inizialmente, sono state studiate varie tecniche di spam detection presenti in letteratura, in modo da identificare quelle più adatte al nostro scopo, e successivamente queste sono state classificate in macro gruppi, sulla base del tipo di segnale utilizzato. Quindi abbiamo cosi diviso le varie tecniche di spam detection in tre gruppi:
\begin{itemize}
 \item tecniche basate sul contenuto;
 \item tecniche basate sul grafo del web;
 \item tecnhiche che usano segnali differenti dalle prime due. 
\end{itemize}
Le prime utilizzano il contenuto di una pagina web \(p\) per determinare se la pagina \(p\) è spam oppure non spam; le seconde utilizzano il grafo delle pagine web, ricavato dai link tra le pagine, per determinare se la pagina è spam o non spam; infine le ultime tecniche usano diversi tipi di segnali per identificare la natura di una pagina web \(p\) ad esempio analizzando alcuni pattern comportamentali dell'utente.\\
Quello che evince da questa fase di analisi è che le tecniche di spam detection basate sul contenuto e sul grafo sono quelle più utilizzate e quelle più studiate. Da notare che le tecnhiche di spam detection basate su grafo sono quelle che meglio riescono a identificare le pagine spam che sfurttano la struttura dei link delle pagine per manipolare gli algoritmi dei motori di ricerca (in particolare quelli che fanno uso della struttura del grafo per definire il rank tra le pagine).\\
Come detto quindi i metodi di spam detection che usano segnali differenti dal contenuto delle pagine o della struttura del grafo sono poco studiati ma questi, come evidenziato dagli studi descritti in precedenza, possono identificare tipi di spam che i metodi classifici non riescono ad identificare ed inoltre alcuni di questi (come quelli che fanno uso di pattern comportamentali) rendono il problema dell’identificazione dello spam scalabile, ovvero riuscire a rilevare nuove tipologie di spam web senza ogni volta definire nuove feature. Questo indica che tali metodi protrebbero essere utilizzati singolarmente avendo delle buone prestazioni. Ma per un identificazine di pagine spam ottimale si consiglia di utilizzare i metodi classifici e metedodi innovativi  in modo complemetare in modo tale da raffinare la rilevazione delle pagine spam.

Dopo la fase di documentazione sui vari metodi di spam detection abbiamo analizzato due algoritmi di spam detection che operano offline, quali \textit{TrustRank} e \textit{Anti-Trust Rank}, durante la fase di crawling. Quindi la fase di analisi ha come obbiettivo la valutazione del comportamento di tali algoritmi in modalità online e determinare se  questi algoritmi possono essere in grado di operare in modalità online, se durante l’esecuzione in modalità online quanto riescono ad approssimare il
a loro comportamento offline e se è conveniente utilizzarli in modalità online.

Dai test che sono stati elaborati si evince che il compartamento di \textit{TrustRank} e \textit{Anti-Trust rank} online approssima abbastanza bene il comportamento offline. Comunque le prestazioni dipendono dalla dimensione del grafo, derivato dalla fase di crawling, su cui vengono eseguiti i test. Ma dai test si evince che entrambi gli algoritmi eseguiti online, anche sul grafo ricavato all'inizio della fase di crawling, riescono ad approssimare abbastanza bene il compartamento offline. Quindi questi algoritmi possono essere usati online.

Il comportamento offline è approssimato molto bene infatti confrontando i due algoritmi durante la fase offline e online si nota che i vettori ricavati durante la fase offline sono molto vicini rispetto ai vettori ricavati durante la l'esecuzione offline. I vettori sono sempre più vicini tanto più il grafo derivato dalla fase di crawling è simile al grafo derivat al termine del crawling. Da notare che i valori delle Tau di Kendall applicate tra i vettori ottenuti in modalità online, nelle prime fasi del crawling, con il vettore ottenuto in modalità offline, sono abbastanza alti; infatti nel \textit{Test numero 1} i valori delle Tau di Kendall,  nell'esame di \textit{TrustRank} ha come valore iniziale 0.7, e nell'esame di \textit{Anti-Trust Rank} ha come valore iniziale 0.75. Questo implica che la distanza tra i due vettori all'inizio del crawling è molto piccola. Perciò tali algoritmi in modalità online appossimano bene il compartamento offline.

Analizzando i due algoritmi evince che \textit{Anti-Trust Rank} approssima fin dall'inizio del crawling meglio il comportamento offline e quindi è più indicato per essere usato durante la fase di crawling. Ma è doveroso evidenziare che dai test risulta che alla fine della visita \textit{Trustrank} tende ad approssimare meglo il compartamento.

Un modo di utilizzare questi algoritmi online sarebbe quello di basarsi sull'isolamento approssimato dell'insieme delle pagine buone. Dal momento che le pagine non spam difficilmente avranno link verso pagine spam, si protrebbero utilizzare gli algoritmi di \textit{TrustRank} e \textit{Anti-Trust Rank} in modalità online per identificare le pagine spam e quindi un volta identificati i nodi spam penalizzare questi e quelli che hanno dei link verso tali nodi. Una soluzione estrema potrebbe essere di elimnare tutti i nodi spam identificati.

Utilizzare questi algoritmi in modalità online quindi risulta essere vantaggioso; sia perché una volta identificate le pagine spam molte altre pagine potrebbero essere non scaricate, ed inoltre si eseguirebbero dei calcoli su grafi molto più piccoli e quindi richiederebbe prestazioni inferiori sia a livello di memoria che computazionali.


